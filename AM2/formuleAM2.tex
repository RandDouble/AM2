\documentclass[a4paper]{article}
\usepackage[T1]{fontenc}
\usepackage[utf8]{inputenc}
\usepackage[italian]{babel}
\usepackage{amsmath}
\usepackage{amssymb}
\usepackage{amsthm}

\newcommand{\numberset}{\mathbb}
\newcommand{\N}{\numberset{N}}
\newcommand{\R}{\numberset{R}}

\theoremstyle{plain}
\newtheorem{Teo}{Teorema}

\begin{document}

\author{Stefano Pilosio}
\title {Cose Utili per AM2}
\maketitle

\section{Integrali Indefiniti}

\section{Sostituzioni utili integrali definiti}

\section{Convergenza Integrali Impropri}

\begin{equation*}
\begin{gathered}
    %caso 1
    \int_0^a{\frac{1}{x^\alpha}\text{d}x}
    \begin{cases}
        \alpha<1   \quad \text{converge}\\
        \alpha\ge1 \quad \text{non converge}
    \end{cases}
    \\
    a \in \mathbb{R} \setminus {0}
    \\
    \\
    %caso 2
    \int_a^\infty{\frac{1}{x^\alpha}\text{d}x}
    \begin{cases}
        \alpha>1   \quad \text{converge}\\
        \alpha\le1 \quad \text{non converge}
    \end{cases}
    \\
    a \in (0, \infty)
    \\
    \\
    %caso 3
    \int_0^a{\frac{1}{x^\alpha\log^{\beta} x}\text{d}x}
    \begin{cases}
        \alpha<1 \quad \forall\beta\quad \text{converge}\\
        \alpha=1 \quad \beta>1     \quad \text{converge}\\
        \alpha=1 \quad \beta\le1   \quad \text{non converge}\\
        \alpha>1 \quad \forall\beta\quad \text{non converge}
    \end{cases}
    \\
    a \in (0,1)
    \\
    \\
    %caso 4
    \int_a^\infty{\frac{1}{x^\alpha\log^{\beta} x}\text{d}x}
    \begin{cases}
        \alpha<1 \quad \forall\beta\quad \text{converge}\\
        \alpha=1 \quad \beta>1     \quad \text{converge}\\
        \alpha=1 \quad \beta\le1   \quad \text{non converge}\\
        \alpha>1 \quad \forall\beta\quad \text{non converge}
    \end{cases}
    \\
    a \in (1, \infty)
    \\
    \\
    %caso 5
    \int_1^a{\frac{1}{\log^{\beta} x}\text{d}x}
    \begin{cases}
        \beta<1   \quad \text{converge}\\
        \beta\ge1 \quad \text{non converge}\\
        \end{cases}
    \\
    a \in (0,1) \vee (1, \infty)
    \\
\end{gathered}
\end{equation*}
\section{Approssimazioni}

\section{Lista Metodi Eq. Differenziali}

\section{Checklist Studi di Funzione}

\section {Ricerca dei massimi e dei minimi}

\subsection{Condizioni Necessarie}

\begin{Teo}[Condizioni Necessarie Estremante Libero]    
    $\\f\colon\Omega\subseteq\R\to\R$, $ x_0\in\Omega^o$, f derivabile in $x_0$.\\
    Se $x_0$ è estremante di f $\implies \, \nabla f(x_0)=\bar{0}$ 
\end{Teo}

Questo non basta in quanto un punto estremante in più dimensioni può essere sia 
un massimo, che un minimo o, infine, un punto di sella 

Nel caso del punto di sella questo non è né un minimo né un massimo, tuttavia è stazionario 

\section{Successioni di funzioni}

\subsection{Dominio della funzione e convergenze}

\begin{enumerate}
    \item Capisci dove è definita la funzione da analizzare per comprendere 
    la continuità della successione
    \item Fai il $\lim_{n \to \infty} f_n(x)$ per capire qual è la funzione limite
    \item Verifica che in tutti i punti si abbia convergenza puntuale alla funzione limite
    \item Verifica la convergenza uniforme con: $\sup_{x\in E}|f_n(x)-f(x)|$,
    questo è eseguito cercando un modo per dimostrare o confutare che in ogni punto che $|f_n-f|
    \to 0$ in un modo o nell'altro. 
\end{enumerate}

\subsection{Continuità}

\begin{Teo} [Condizioni per la continuità] 
    Se ogni $f_n$ è continua e la successione è convergente uniformemente $\implies$ $f$ è continua 
    in $x_0$, in particolare se $f_n$ è continua su E, $f$ è continua in E.\\
    \textbf{OSSERVAZIONE} : se $f_n$ è continua uniforme in E e ha la convergenza uniforme in E
    $\implies$ $f$ è continua uniforme. 
\end{Teo}

\subsection{Differenziabilità}

\begin{Teo}[Condizioni per la derivata]
    Sia $J\subseteq\R$, intervallo, $\{f_n\}_{n\in\N}$ la successione di funzioni.
    Se sono verificate le seguenti condizioni:
    \begin{itemize}
        \item $\{f_n\}_{n\in\N}$ converge puntualmente in $J$ alla funzione limite $f$.
        \item Ogni $f_n$ è derivabile in ogni punto di $J$.
        \item $\{f'_n\}_{n\in\N}$ converge uniformemente su ogni intervallo limitato $I\subseteq J$
    \end{itemize}
    $\implies$ f è derivabile su J e $\forall x \in J$ si ha 
    \begin{equation*} 
        f'(x)=\lim_{n\to\infty}{f'_n(x)}
    \end{equation*} 

\end{Teo}

\section {Serie di Funzioni}

\subsection{Dominio della funzione e convergenze}

\begin{enumerate}
    \item Comprendi dove $f_n$ è definita anche secondo le condizioni del problema,
    se anche una sola delle funzioni non esiste allora non può esistere la serie in quel punto.
    \item Studia la convergenza della serie come in analisi 1, questa convergenza è solo puntuale.
    \item Per la convergenza uniforme studia $sup_{x\in E}{|f_n(x)|}$. Se questo ha un massimo
    che non tende a zero all'aumentare di n allora non si ha convergenza uniforme.
\end{enumerate}

\textbf{OSSERVAZIONE}: In caso l'ultimo punto fallisse, c'è un ulteriore metodo, che sfrutta il criterio
di Weierstrass, che prevede di maggiorare $|f_n|$ con una successione numerica $M_n$ convergente, questo garantisce 
convergenza più forte.

\subsection{Conseguenze convergenza uniforme}
\begin{Teo}{Proprietà della convergenza uniforme}
    Se ho una serie di funzioni convergente su $E\subseteq\R$ alla funzione somma $S$, 
    ho le seguenti proprietà:
    \begin{itemize}
        \item Se ogni $f_n$ è limitata su E $\implies S$ è limitata su $E$.
        \item Se ogni $f_n$ è continua in $x_0\in E\implies S$ è continua in $x_0$.
        \item Se ogni $f_n$ è continua su E $\implies S$ è limitata su $E$.
        \item Se ogni $f_n$ è uniformemente continua in E $\implies S$ è uniformemente continua su $E$.
    \end{itemize}
\end{Teo}
\begin{Teo}{Integrale e Sommatoria}
    Se $f_n$ è continua in [a,b] e la sua serie converge uniformemente sull'intervallo a S.
    $$\implies \int_a^b{S }=\int_a^b{\sum_{n=0}^\infty{f_n(x)}dx}=\sum_{n=0}^\infty{\int_a^b{f_n(x)dx}}$$
\end{Teo}

\section {Equazioni differenziali di primo ordine}
\subsection{Esistenza e Unicità}
\begin{Teo}[Locale]
    
\end{Teo}
\begin{Teo}[globale]
    
\end{Teo}

\subsection{a Variabili Separabili}

Equazione della forma:
\begin{equation*}
    y'=h(x)\cdot k(y)
\end{equation*}
Soluzione:
\begin{gather*}
    \int_{x_0}^x{\frac{y'(x)}{k(y(x))}\text{d}x}=\int_{x_0}^{x}{h(x)\text{d}x}\\
    \text{sostituisco}\quad u=y(x),\, du=y'(x)dx\\
    \int_{y_0}^y{\frac{\text{d}u}{k(u)}}=\int_{x_0}^{x}{h(x)\text{d}x}
\end{gather*}

\subsection{Lineari del primo ordine}

Equazione della forma:
\begin{equation*}
    y'+p(x)y=q(x),\quad p,q\colon J\to\R,\,J\subseteq\R,\, p,q\text{ continue in }J
\end{equation*}
Soluzione:
\begin{gather*}
    y(x)=\exp \left( -\int_{x_0}^x{p(t)\text{d}t} \right) \cdot
    \left\{ y_0+\int_{x_0}^{x}{q(r)\exp \left(\int_{x_0}^r{p(t)\text{d}t}\right) \text{d}r} \right\}
\end{gather*}

\subsection{Omogenee}

Equazione della forma:
\begin{equation*}
    y'=g\left(\frac{y}{x}\right)    
\end{equation*}
Soluzione: 
\begin{gather*}
    \text{Sostituisco }t=\frac{y}{x}\\
    \implies y=x\cdot t(x),\,y'(x)=t(x)+x\cdot t'(x)\\
    \implies t(x)+x\cdot t'(x)=g(t(x)),\,\text{si risolve con le variabili separabili}
\end{gather*}

\end{document}