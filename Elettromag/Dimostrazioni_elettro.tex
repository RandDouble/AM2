\documentclass[a4paper]{scrarticle}

\usepackage[T1]{fontenc}
\usepackage[utf8]{inputenc}
\usepackage[italian]{babel}

\usepackage{amsmath, amsthm, amsfonts}
\usepackage{bm}
\usepackage{array}
\usepackage{physics}
\usepackage{rotating}
\usepackage{booktabs}

\begin{document}


\title{Appunti di Elettromagnetismo (Basato su Ragusa)}
\author{Stefano Pilosio}

\maketitle

\section{Funzione $\delta(x)$ e Sorgente Puntiforme}

\subsection{Caratteristiche $\delta(x)$}

La funzione $\delta(x)$ è un funzionale (manda una funzione in un numero), non una funzione, quindi non è definita fuori dal segno di integrale.

è rappresentata come una funzione che vale zero ovunque, tranne in un punto in cui vale infinito.

Il suo comportamento è:
\begin{equation*}
    \int_{a}^{b} f(x) \delta (x-x_0) dx = f(x_0)
\end{equation*}

La funzione $\delta$ ha le dimensioni di una densità

Si puo definire $\delta^3(\vec r - \vec r_0) = \delta(x-x_0)\delta(y-y_0)\delta(z-z_0)$

\subsection{Dimostrazione Sorgente Puntiforme}

Considero $\bm {F} (\bm r) = \frac{\hat {\bm{r}}}{r^2}$, tipico per del campo elettrico descritto dalla forza di Coulomb.

Se calcolo la divergenza di $\bm F$, questa è nulla.
Se calcolo il suo flusso ottengo: $\oint_{\Sigma} \bm {F} \cdot d \bm{a} = 4\pi$

Sfrutto il teorema della divergenza:
\begin{equation*}
    \oint_{V} \div{\bm{F}} d\tau = \oint_{\Sigma} \bm{F} \cdot d \bm{a} = 4\pi
\end{equation*}

Da cui ottengo:

\begin{equation*}
    \int_{V_sfera} \div {\bm F} dV = 4\pi
\end{equation*}

Ho una funzione nulla che però ha integrale non nullo:
nel caso di sorgente puntiforme $\div {\bm F} = 0$, tranne nel punto dove si trova la sorgente, in quel punto vale $\infty$, inoltre il suo integrale è diverso da zero. Si comporta alla stessa maniera di un funzionale: la $\delta(x)$:
Quindi ottengo:
\begin{equation*}
    \div{\bm{F}} = 4\pi \delta^3(\bm r)
\end{equation*}

Infine per le ipotesi sappiamo che:
\begin{equation*}
    \bm F (\bm r) = \frac{\hat r}{r^2} = - \div \frac{1}{r} \implies \laplacian \frac{1}{r} = - 4 \pi \delta ^3 (\bm r)
\end{equation*}

\subsection{Conseguenze}

Posso rappresentare una carica puntiforme mediante una distribuzione $q \to \rho(\bm r) = q\delta^3(\bm r - \bm r')$

Questa produce il risultato corretto:

\begin{gather*}
    \bm E (\bm r) = \frac{1}{4\pi\varepsilon_0}\int_{V} \frac{q \delta^3 (\bm r' - \bm r_0)}{\left|\bm r - \bm r'\right|^2}\frac{\bm r - \bm r'}{\left| \bm r - \bm r '\right| d^3 \bm r'} = \frac{1}{4 \pi \varepsilon_0} \frac{q}{\left|\bm r - \bm r'\right|^2} \frac{\bm r - \bm r_0}{\left|\bm r - \bm r_0\right|}
\end{gather*}

Il potenziale con questa forma soddisfa l'equazione di Poisson, basta buttarlo dentro l'equazione per verificarlo.
\begin{equation}
    \phi_p(\bm r) = \frac{1}{4\pi\varepsilon_0}\int_{V}\frac{\rho(\bm r') d^3\bm r'}{\left|\bm r - \bm r'\right|}
\end{equation}

\section{Condensatori}

I condensatori sono dispositivi che possono immagazzinare energia sotto forma di un campo elettrico.

Per definizione la carica sulle armature è $Q = C \Delta V $
Se trasporto una carica dall'armatura superiore (che passa da $q$ a $q - dq$) a quella inferiore (che passa da $q$ a $q + dq$) devo compiere un lavoro:

\begin{gather*}
    dW = V dq\\
    q = CV\implies V = q/C \\
    dW = \frac{q}{C} dq    
\end{gather*}

Integrando:
\begin{equation}
    W = \int_{0}^{Q} \frac{q}{C} dq = \frac{1}{2}\frac{Q^2}{C}
\end{equation}

Il lavoro compiuto è uguale all'energia immagazzinata ($U  = W$), questo lavoro può essere riespresso in funzione del potenziale:

\begin{gather*}
    U = \frac{1}{2} \frac{Q^2}{C} = \frac{1}{2} CV^2
\end{gather*}

Nel condensatore trascurando gli effetti di bordo ho un campo elettrico uniforme diretto dall'armatura a potenziale maggiore all'armatura a potenziale minore.
Il modulo di $\bm E$ è : $E \frac{\sigma}{\varepsilon_0}$, la densità di energia $\rho_E = \frac{1}{2}\frac{\sigma^2}{\varepsilon_0}$, per ottenere l'energia basta fare l'integrale sul volume del condensatore.

\subsection{Forza fra le armature di un condensatore}

Condensatore carico, sulle armature ho $\pm Q$
Ho le armature che si attraggono, ho una forza non elettrica che mantiene la distanza fra le armature.

Primo conto considerando condensatore carico, ma isolato da un generatore, innalzo di un tratto $dz$ l'armatura superiore, la forza meccanica applicata è bilanciata dalla forza elettrica.
\begin{equation*}
    dW = \bm F_m \cdot d\bm r = F_m dz = - F_e dz
\end{equation*}
In questo modo anche l'energia immagazzinata aumenta, in quanto la capacità diminuisce:
\begin{equation*}
    U = \frac{1}{2}\frac{Q^2}{C} \implies dU = \frac{1}{2}Q^2 d \frac{1}{C} = \frac{1}{2}\frac{Q^2}{C}\frac{dz}{L}
\end{equation*}

L'aumento di energia è dovuto al lavoro fatto dalla forza meccanica. Uguagliando le due espressioni ottengo la forza elettrica.
Vedo che le due armature di attraggono.

Se la differenza di potenziale è tenuta costante quello che succede è che le cariche varieranno, quindi ho sia un lavoro per muovere l'armatura, che un lavoro per muovere le cariche $dW_G = VdQ = V^2 dC$:
Il bilancio energetico corretto diventa:
\begin{gather*}
    dW_m + dW_G = dU \implies -F_edz + V^2dC = \frac{1}{2}V^2dC\\
    \to -F_e dz = -\frac{1}{2}V^2dC\\
    \to F_e = -\frac{1}{2}\frac{Q^2}{C}\frac{1}{L}
\end{gather*}

è abbastanza analogo il problema in cui le facce si spostano parallelamente l'una all'altra.

\subsection{Conduttori}

\emph{Si parla di induzione completa quando tutte le linee del campo elettrico passano attraverso il conduttore (Caso del condensatore in cui si trascurano gli effetti di bordo)}
Se voglio una relazione per determinare la carica su ogni conduttore noti i potenziali sui conduttori stessi devo usare il teorema di unicità dell'equazione di Poisson.

\begin{itemize}
    \item Metto a potenziale $ \phi = 0 $ tutti i conduttori meno il primo, che rimane a $\phi_1$
    \item Per effetto di $\phi_1$ sugli altri conduttori compare una carica elettrica $q_1, \dots, q_n$
    \item C'è una relazione lineare tra potenziale e le cariche: $q_i = C_{i1} \phi_1$
    \item ripeto per ogni conduttore, spegnendo ogni altro conduttore tranne quello in analisi
    \item Sfruttando al linearità la carica finale su ogni conduttore è 
    \begin{equation}
        q_i = \sum_{k=1}^{n}C_{ik} \phi_k
    \end{equation}
\end{itemize}

Si può dimostrare che i i coefficenti $C_{ik}$ sono simmetrici sotto scambio di indice, e sono detti \emph{coefficienti di capacità}.
Posso invertire le relazioni per trovare i potenziale, nel nuovo caso i coefficienti della matrice inversa sono detti \emph{coefficienti di potenziale}.


\section {Dielettrici}

\subsection{Dipolo Elettrico}

Sistema di due cariche di segno opposto $-q$ e $+q$ poste a distanza $d$.

Si parte calcolando il potenziale:
\begin{equation*}
    \psi = \frac{q}{4\pi\varepsilon_0} \frac{1}{\left|\bm r - \bm r_+\right|} - frac{q}{4\pi\varepsilon_0} \frac{1}{\left|\bm r - \bm r_-\right|}
\end{equation*}

Questa formula vale a qualunque distanza, ma a noi interessa a distanze maggiori rispetto a quella dell'atomo ($r \gg d$). Mi serve una formula approssimata.

Considero il potenziale in un punto $\bm r$, uso le coordinate sferiche (Ho simmetria per rotazioni attorno a z, quindi non dipendo dall'angolo azimutale).

Analisi dei denominatori:
\begin{equation*}
    \left|\bm r - \bm r_+\right|= \sqrt{r^2 + \frac{d^2}{4} - rd \cos \theta} = r \sqrt{1 + \frac{d^2}{4r^2} - \frac{d}{r}\cos \theta}
\end{equation*}

Per le condizioni poste in precedenza posso ignorare il termine al quadrato:
\begin{equation*}
    \frac{d^2}{r^2} \ll 1
\end{equation*}

Quindi ottengo:
\begin{gather}
    \frac{1}{\left|\bm r - \bm r_+\right|} \approx \frac{1}{r \sqrt{1 - \frac{d}{r}\cos \theta}} \approx \frac{1}{r}(1 + \frac{d}{2r}\cos \theta)\\
    \frac{1}{\left|\bm r - \bm r_-\right|} \approx \frac{1}{r \sqrt{1 + \frac{d}{r}\cos \theta}} \approx \frac{1}{r}(1 - \frac{d}{2r}\cos \theta)
\end{gather}

Introduco le approssimazioni nel potenziale:

\begin{gather*}
    \psi \approx \frac{q}{4\pi\varepsilon_0}\left[\frac{1}{r}(1 + \frac{d}{2r}\cos \theta) - \frac{1}{r}(1 - \frac{d}{2r}\cos \theta)\right] \quad \text{I termini uguali si elidono}\\
    \psi \approx \left[\frac{d}{2r^2}\cos \theta + \frac{d}{2r^2}\cos \theta\right]
\end{gather*}

\begin{equation}
    \psi = \frac{qd}{4\pi\varepsilon_0}\frac{\cos\theta}{r^2}
\end{equation}

Dipende solo dal prodotto $\bm p = q \bm d$, che è detto \emph{Momento di Dipolo} (Per le approssimazioni adottate il dipolo ideale è puntiforme).
Il potenziale varia come $1/r^2$, quindi si attenua più velocemente di una carica puntiforme.
Ho una dipendenza rispetto a $\theta$, quindi perdo la simmetria sferica, ho una direzione privilegiata che è l'asse delle due cariche, che in questo caso è l'asse z.

Usando una scrittura vettoriale posso riscrivere come:

\begin{equation}
    \psi(r, \theta) = \frac{1}{4 \pi \varepsilon_0} \frac{\bm{\hat e_r} \cdot \bm p}{r^2}
\end{equation}

Il campo elettrico diventa:
\begin{equation}
    \bm E = \frac{1}{4\pi\varepsilon_0} \frac{p}{r^3} (2 \va{ e_r}\cos \theta + \va{ e_\theta} \sin\theta )
\end{equation}

\subsubsection*{Forza Sul Dipolo}

Partendo dal caso di campo elettrico uniforme

Il sistema dipolo in se e per se non trasla, infatti essendo composto di due cariche di segno opposto su di esse agiscono due forze di uguale intensità e di segno opposto, quindi si cancellano. Tuttavia è perfettamente libero di ruotare:
\begin{equation}
    \bm \tau = q \bm E \times \bm d = \bm E \times \bm p
\end{equation}

Quindi ho un lavoro di rotazione che si ottiene intergando la seguente:
\begin{equation*}
    dW = \tau d\alpha
\end{equation*}

Come sempre il lavoro della forza esterna è pari a una variazione dell'energia interna del sistema, quindi ottegno che:
\begin{equation*}
    U(\theta) = -pE\cos \theta = - \bm p \cdot \bm E
\end{equation*}

In caso di campo elettrico non uniforme invece succedono cose, ma tutto si riassume in:

\begin{equation*}
    \bm F = (\bm p \cdot \nabla)\bm E
\end{equation*}

\section{Materia Polarizzata}

\subsection{Campo elettrico generato}

Introduco $\bm P = n(\bm r)\bm p$ \emph{Vettore di densità di polarizzazione}.
Con $n(\bm r)$ che è la densità di dipoli per unità di volume.
Quindi la sua unità di misura è \emph{Coulomb al metro quadro}.
Suppongo di avere un volume infinitesimo con un numero di dipoli molto grade, allora posso definire:
\begin{equation*}
    d\bm p = \bm P dv
\end{equation*}

Per calcolare il campo elettrico all'esterno suddivido il blocco in tante colonne verticali, a questo punto calcolo il campo elettrico generato da ogni colonna per cui $dv = da dz$, sfrutto il potenziale generato da un dipolo:
\begin{align*}
    d\phi(\bm r) &= \frac{d\bm p \cdot \va r}{4\pi\varepsilon_0r^2}= \frac{dp\cos\theta}{4\pi\varepsilon_0r^2}\\
    &= \int_{z_1}^{z_2}\frac{Pdadz\cos \theta}{4\pi\varepsilon_0}\\
    &= \frac{Pda}{4\pi\varepsilon_0}\int_{z_1}^{z_2}\frac{dz \cos \theta}{r^2}
\end{align*}

è necessario guardare la relazione tra $dr$ e $dx$:
\begin{equation*}
    dz \cos \theta = -dr
\end{equation*}

\begin{align*}
    d\phi(\bm r) &= \frac{Pda}{4\pi\varepsilon_0} \int_{z_1}^{z_2}\frac{dz\cos\theta}{r^2}\\
    &= \frac{Pda}{4\pi\varepsilon_0} \int_{r_1}^{r_2} - \frac{dr}{r^2}\\
    &= \frac{Pda}{4\pi\varepsilon_0} (\frac{1}{r_2}- \frac{1}{r_1})
\end{align*}

Questa relazione è identica  aquella del potenziale generato da una carica $+Pda$ posta in $z_2$ e una carica $-Pda$ posta in $z_1$.
Per ottenere il potenziale a questo punto basta fare il flusso attraverso la superficie del dielettrico (In quanto il campo che esce dalla superificie laterale di una colonna, entra in quella laterale di un'altra colonna, a meno che non ci si trovi sulla superifice del blocco).
Il risultato finale è che il blocco di materiale polarizzato generea un potenziale elettrico identico a quello di due densità di carica superificiale poste sulle superfici esterne del blocco.

Da quest'ultima considerazione si ottiene:
\begin{equation}
    \sigma  = P = |\bm P|
\end{equation}

In questo caso si è assunta polarizzazione uniforme e diretta lungo l'asse z.


\section{Forza di Lorentz}

La forza totale agente su una carica è:
\begin{equation}
    \bm F = q \left(\bm E + \bm v \times \bm B\right)
\end{equation}

Questa vale anche per campi variabili nel tempo, è una relazione locale (tutte le grandezze sono misurate nello stesso punto allo stesso tempo e rispetto allo stesso sistema di riferimento inerziale).
La forza di Lorentz è una forza che non compie lavoro essendo sempre perpendicolare alla traiettoria.
Quindi una particella che si muove con velocità uniforme percorre un \emph{Moto Circolare Uniforme}, caratterizzato da una frequenza detta \emph{di ciclotrone}:
\begin{equation}
    \omega = \frac{qB}{m}
\end{equation}

\subsection*{Forza su un filo percorso da corrente}

Filo conduttore di sezione $S$ percorso da una corrente $I$, i portatori portano una carica $q$ e ho una densità $\rho_N$ di portatori nel filo.
Suppongo che tutte le particelle abbiano la stessa velocità $v$, se prendo un tratto infinitesimo di filo $d\bm l $ trovo la seguente relazione:
\begin{equation*}
    dN = \rho_N S |d\bm l|
\end{equation*}

Per il numero di cariche all'interno del filo conduttore.

Se sono immerso in un campo magnetico su ogni particella agisce la forza di lorentz, che sul tratto si traduce in:

\begin{gather*}
    d \bm F = \bm f  dN\\
    \bm f = q \bm v \times \bm B\\
    d \bm F = q \bm v \times \bm B  dN\\
    d \bm F = q \rho_N S |d\bm l| \bm v \times \bm B\\
    \bm I \parallel |d \bm l| \implies \bm v \parallel |d \bm l|\\
    |\bm J| = q \rho_N |\bm v|\\ 
    d \bm F = |\bm J| S |d\bm l| \times \bm B\\
    d \bm F = I |d\bm l| \times \bm B\\
    \bm F = I \oint_C d\bm l \times \bm B
\end{gather*}

Ho trovato la forza totale agente su un circuito $C$.
Posso fare il discorso precedenti con i plasmi oppure  con cariche in un fluido parlando della densità volumetrica di corrente e della forza volumetrica.

\subsection{Legge di Biot-Savart}
\subsubsection*{Sorgenti del campo magnetico}

Operativametne il campo magnetico è generato da una carica in movimento. La forza di Lorentz non definisce chi sia la sorgente del campo magnetico. Abbiamo i magneti permanenti che generano un campo magnetico, abbiamo le corretni che subiscono un campo magnetico e ne generano uno. La vera origine dei campi magnetici è complessa, serve una teoria quantistica.

\subsubsection*{La legge}

Detta anche prima formula di Laplace, permette il calcolo del campo di induzione magnetica generato da un filo percorso da corrente. \emph{Vale solo per correnti stazionarie}. 
Considero filo percorso da una corrente $i$. Considero un elemento $dl$ del filo. Il contributo al campo $\bm B$ inel punto $\bm r$ è dato da:
\begin{equation}
d \bm B = \frac{\mu_0}{4\pi}i \frac{d \bm l \times \va r}{r^2}
\end{equation}

Il pezzo di filo fa parte di un circuito chiuso, bisogna integrare su ogni pezzo di filo per ottenere il campo magnetico finale:
\begin{equation}
    d \bm B(\bm r_2) = \frac{\mu_0}{4\pi} i \frac{d \bm l_1 \times (\bm r_2 -\bm r_1)}{\left|\bm r_2 - \bm r_1\right|^3}
    \implies \bm B (\bm r_2) = \frac{\mu_0}{4 \pi}i \oint_{filo} \frac{d \bm l_1 \times (\bm r_2 -\bm r_1)}{\left|\bm r_2 - \bm r_1\right|^3}
\end{equation}

\subsubsection*{Conduzione di Magnetostatica}

La condizione di stazionarietà che definisce la magnetostatica è dovuta a delle cariche in movimento, un movimento con delle caratteristiche precise, per cui non si hanno variazioni sulla densità di carica: Deve valere la \emph{condizione di continuità}

\begin{equation}
    \div \bm J = - \pdv{\rho}{t}
\end{equation}

In particolare per la magnetostatica deve valere che il termine a sinistra sia nullo, quindi si riduce a:
\begin{equation}
    \div \bm J = 0
\end{equation}

Per cui si hanno correnti continue, ossia che non variano nel tempo.

\subsection{Campo Magnetico Filo}

Filo infinito lungo l'asse $x\implies d \bm l = \va e_x dx$, il punto su cui calcolare il campo magnetico giace sull'asse $y$.
\begin{gather*}
    \va r = \va e_x \sin \theta + \va e_y \cos \theta\\
    \va e_x \times \va e_x = 0\\
    d \bm l \times \va r = \va e_x \times \va e_y \sin \theta dx = \va e_z \sin \theta dx
\end{gather*}
Il contributo del campo magnetico è quindi perpendicolare al piano xy. Ora voglio far dipendere tutto dall'angolo $\theta$, visto che integrare su tutti i reali fa un po' schifo\dots

\begin{gather*}
    (\text{a distanza verticale di r da dl})\, a = r \sin(\pi - \theta) = r \sin \theta \implies r = \frac{a}{\sin \theta}\\
    (\text{x distanza orizontale di r da dl})\, a= x \tan (\pi -\theta) = -x \tan \theta \implies x = -a \frac{\cos \theta}{\sin \theta}\\
    \implies dx = a \frac{d \theta}{\sin^2 \theta}\\
    d \bm B = \frac{\mu_0}{4\pi}i \frac{\va e_z \sin \theta}{a}d\theta
\end{gather*}

Integrando da $0 \to \pi$ ho ottenuto il risultato cercato.

\section{Relatività}

Supponiamo una carica che inizi a muoversi a $t=0$, con velocità uniforme $\bm b$, ho delle variazioni del campo elettrico dovute solo al fatto che adesso la carica è in moto. Queste variazioni \emph{non possono essere percepite istantaneamente in tutto lo spazio}. La perturbazione indotta deve viaggiare con \emph{la velocità della luce} $c$.
Quindi al tempo $t$ i punti nello spazio a distanza superiore a  $R=ct$, non possono sapere che la carica si sia messa in moto, quelli a distanza inferiore invece vedono la carica in moto. Ho due campi elettrici separati da una zona di transizione, il cui spessore dipende dal tempo che la particella ha impiegato a passare dallo stati di riposo al moto rettilineo uniforme.
Nella regione di transizione passiamo da un campo radiale  a un campo perpendicolare al raggio. Questo è analogo se la carica si arresta.

Si può verificare partendo dalla geometria nel caso della carica che si ferma che il raccordo delle linee di campo segue la seguente legge:
\begin{equation*}
    \tan \phi_0 = \gamma \tan \theta_0
\end{equation*}

\subsection{quadrivettore}

Nel passaggio da un sistema di riferimento ad un altro la posizione e il tempo di un evento si trasformano in una maniera non banale:
\begin{itemize}
    \item Lo spazio e il Tempo non sono quantità indipendenti
    \item Sono le componenti del quadrivettore posizione $(ct,x,y,z)$
    \item Nuova notazione: cambio nome alle coordinate spaziali $(x,y,z) \to (x^1, x^2, x^3)$
    \item introduco una coordinata temporale: $x^0 = ct$
    \item introduco $\beta= \frac{v}{c}$
    \item introduco $\gamma = \frac{1}{\sqrt{1 - \beta^2}}$
\end{itemize}

A questo punto scrivo le trasformazioni di Lorentz come:
\begin{equation}
    \begin{cases}
        x' = \gamma (x - \beta ct)\\
        y' = y\\
        z' = z\\
        ct' = \gamma(ct - \beta x)
    \end{cases}
\end{equation}

O in forma matriciale:
\begin{equation}
    \begin{pmatrix}
        x'^0 \\ x'^1\\x'^2\\x'^3
    \end{pmatrix}
    =
    \begin{pmatrix}
        \gamma & -\gamma \beta & 0 & 0 \\
        - \gamma \beta & \gamma & 0 & 0 \\
        0 & 0 & 1 & 0 \\
        0 & 0 & 0 & 1
    \end{pmatrix}
    \begin{pmatrix}
        x^0 \\ x^1\\x^2\\x^3
    \end{pmatrix}
\end{equation}

Per costruzione questa lasciata invariata la quantità detta \emph{invariante relativistico}
\begin{equation}
    (x^0)^2 - (x^1)^2 - (x^2)^2 - (x^3)^2 
\end{equation}

Se si applica questa operazione alle quantità trasformate si ottiene che la condizione di uguaglianza si ha se:
\begin{equation}
    \gamma^2 (1 - \beta^2) = 1
\end{equation}

Questo è facile da verificare, inoltre si nota che questo è il determinante della matrice della trasformazione di Lorentz.

Quelle in precedenza definite sono le componenti controvarianti del quadrivettore posizione, queste hanno indice alto.

Posso definire anche le componenti covarianti, queste si differenziano per l'indice basso, e per la parte spaziale con segno invertito:
\begin{equation}
    x_\mu = (ct , -x,-y,-z)
\end{equation}

Inoltre per calcolare l'invariante posso definire il modulo quadrato el quadrivettore come:
\begin{equation}
    x \cdot x = x^\mu x_\mu
\end{equation}

\section{Fem Indotta}


Spiega la terza equazione di Maxwell, ricavata da Micheal Faraday nel 1830 in tre diversi esperimenti: 
\begin{equation}
    \mathcal{E} = -\frac{d \varPhi}{dt}
\end{equation}

Caratteristiche degli esperimenti:
\begin{itemize}
    \item Campo magnetico fisso, spira che si muove
    \item Campo magnetico che si muove, spira ferma (Uguale al primo, ma rispetto a un altro sistema di riferimento)
    \item Campo magnetico variabile nel tempo
\end{itemize}

Presa una curva chiusa e stazionaria $\mathcal{C}$ in un sistema inerziale $\mathcal{S}$. La curva delimita una superficie $\mathcal{A}$. E' presente un campo magnetico dipendente dal tempo.

Dalla legge di Faraday si ottiene:
\begin{gather*}
    \oint_\mathcal{C} \bm E \cdot d\bm l = - \frac{d \varPhi}{dt} = - \frac{d}{dt} \int_A \bm B \cdot d\bm a\\
    text{Uso il teorema di Stokes}\\
    \int_A (\curl \bm E)\cdot d\bm a = - \frac{d}{dt} \int_A \bm B \cdot d\bm a\\
    \text{sia A che C sono stazionarie}\\
    \int_A (\curl \bm E)\cdot d\bm a = -  \int_A \frac{d\bm B}{dt} \cdot d\bm a\\
    \text{La curva è arbitraria}\\
    \curl \bm E = - pdv{\bm B}{t}
\end{gather*}

Ho ottenuto la terza equazione di MAxwell per il terzo caso.

\subsubsection*{Conseguenze}

Valgono sempre la terza e la seconda equazione di maxwell, dalla seconda si deriva che per $\bm B$ posso definire un potenziale vettore $\bm A$, tale che $\bm B = \curl \bm A$.

Se si inserisce questa relazione nella terza equazione di Maxwell si ottiene:
\begin{equation}
    \curl \bm E = - \pdv{\curl \bm A}{t} = -\curl \pdv{\bm A}{t}\implies \curl \left(\bm E + \pdv{\bm A}{t}\right)= 0
\end{equation}

Quindi ho trovato una combinazione di campi a rotore nullo, questo vuol dire che è la combinazione è il gradiente di un campo scalare:
\begin{equation}
    E + \pdv{\bm A}{t} = - \nabla \phi
\end{equation}

Da cui trovo l'espressione corretta per il campo elettrico:
\begin{equation}
    \bm E = - \nabla \phi - \pdv{\bm A}{t}
\end{equation}

\subsubsection*{Auto Induttanza}

Se considera una spira percorsa da corrente $I$, la spira genera un campo magnetico $\bm B$. Sfrutto la legge di Biot-Savart:
\begin{equation*}
    \bm B(\bm r) = \frac{\mu_0}{4\pi} I \oint_{\mathcal{C}} \frac{d\bm l_1 \times (\bm r - \bm r_1)}{\left|\bm r - \bm r_1\right|^3}
\end{equation*}

Il campo magnetico dipende linearmente dalla corrente, a questo punto calcolo il flusso del campo magnetico inserendo l'equazione di Biot-Savart, di solito è una cosa difficile da calcolare come integrale, tuttavia la cosa importante è che questo è ancora proporzionale alla corrente, quindi posso riscriverlo come:
\begin{equation}
    \varPhi = LI
\end{equation}

Da cui ricavo che L è:
\begin{equation}
    L = \frac{\mu_0}{4\pi} \int \left[\frac{d \bm l_1 \times (\bm r - \bm r')}{|\bm r - \bm r_1|^3}\right] \cdot d\bm a
\end{equation}

Questa forma si può ottenere in una formula più semplice a partire dalla definizione di flusso e dal potenziale vettore:
\begin{equation*}
    \varPhi = \int_S \bm B \cdot d \bm a = \int_S \curl \bm A \cdot d \bm a = \oint_{\mathcal{C}} \bm A \cdot d \bm l
\end{equation*}

Sfruttando l'espressione per il potenziale vettore:

\begin{equation}
    \bm A (\bm r) = \frac{\mu_0}{4\pi} I \oint_{\mathcal{C}} \frac{d \bm l'}{|\bm r - \bm r'|}
\end{equation}

Questa introdotta nella definizione di flusso ottengo che L è:
\begin{equation}
    L = \frac{\mu_0}{4\pi} \oint_{\mathcal{C}}\oint_{\mathcal{C}} \frac{d\bm l\cdot d\bm l}{|\bm r - \bm r'|}
\end{equation}

Il discorso con la mutua Induttanza è legato al flusso del secondo circuito attraversato dal campo magnetico del primo circuito.


\section{Onde Elettromagnetiche}

\subsection{Teorema di Poynting}

Energia associata a campi elettrici:
\begin{equation*}
    U_E = \frac{\varepsilon_0}{2} \int E^2 dV 
\end{equation*}

Energia associata a campi magnetici:
\begin{equation*}
    \frac{1}{2 \mu_0} \int B^2 dV
\end{equation*}

Queste valgono anche per campi elettromagnetici, faccio semplicemente la somma:

\begin{equation}
    U_{EM} = \frac{1}{2} \int (\varepsilon_0 E^2 + \frac{1}{\mu_0} B^2)dV
\end{equation}

Suppongo di avere una distribuzione di carica $\rho$ e di corrente $\bm J$, che generino un campo EM.

Suppongo le cariche in movimento, $\bm v$ è la velocità della carica contenuta in $dV$.
Calcoliamo il lavoro fatto dal campo EM sulle cariche e le correnti nel tempo $dt$:
Prendo un eletento di volume $dV$,
\begin{equation*}
    dq = \rho dV \quad \bm J dV = \rho \bm v dV = dq \bm v
\end{equation*}

Sulla carica infinitesima agisce la Forza di Lorentz:

\begin{gather*}
    dw dV = \bm f \cdot d\bm s = \bm f \cdot \bm v dt = dq (\bm E + \bm v\times \bm B )\cdot \bm v dt = dq \bm E \cdot \bm v dt\\
    dw dV = \rho dV \bm E \cdot \bm v dt = \rho \bm v \cdot \bm E dtdV = \bm J \cdot \bm E dtdV
\end{gather*}

Quindi la potenza erogata nel volume è:
\begin{equation}
    \frac{dw}{dt}dV = \bm J \cdot \bm E dV
\end{equation}

Integrando sul Volume:

\begin{equation*}
    \frac{dW}{dt}= \int \bm E \cdot \bm J dV
\end{equation*}

Considero la IV equazione di Maxwell:

\begin{equation*}
    \curl {\bm B} = \mu_0 \bm J + \mu_0\varepsilon_0 \pdv{\bm E}{t} \implies \bm J = \frac{1}{\mu_0} \curl {\bm B} - \varepsilon_0 \pdv{\bm E}{t}
\end{equation*}

La inserisco all'interno dell'integrale:

\begin{equation*}
    \frac{dW}{dt} = \int \left(\frac{1}{\mu_0} \bm E \cdot \curl{\bm B} - \varepsilon_0 \bm E \cdot \pdv{\bm E}{t} \right) dV
\end{equation*}

Sfrutta la regola del prodotto scalare per il rotore:
\begin{gather*}
    \div (\bm A \times \bm C) = \bm C \cdot (\curl{\bm A}) - \bm A \cdot (\curl{\bm C})\\
    E \cdot \curl{\bm B} = - \div{\bm E \times \bm B} + \bm B \cdot \curl{\bm E}\\
    \curl{\bm E} = - \pdv{\bm B}{t}\\
    E \cdot \curl{\bm B} = - \div{\bm E \times \bm B} - \bm B \cdot \pdv{\bm B}{t}
\end{gather*}

Inserita nell'espressione della potenza:

\begin{equation*}
    \frac{dW}{dt} = \int \left(- \frac{1}{\mu_0}\div{\bm E \times \bm B} - \bm B \cdot \pdv{\bm B}{t} - \varepsilon_0 \bm E \cdot \pdv{\bm E}{t}\right)dV
\end{equation*}

Si può osservare che:
\begin{equation*}
    \bm E \cdot \pdv{E}{t} = \frac{1}{2} \pdv{E^2}{t} \qquad
    \bm B \cdot \pdv{B}{t} = \frac{1}{2} \pdv{B^2}{t}
\end{equation*}

Quindi posso sostituire all'intero dell'integrale:
\begin{equation*}
    \frac{dW}{dt} = -\frac{1}{\mu_0}\int \div{\left(\bm E \times \bm B\right)} dV - \pdv{t}\frac{1}{2}\int\left(\frac{1}{\mu_0}B^2- \varepsilon_0 {E^2}\right) dV
\end{equation*}

Il secondo integrale descrive l'energia elettromagnetica del sistema, invece il primo integrale descrive l'energia che fluisce attraverso la superficie che delimita il sistema.
Sfrutto il teorema della divergenza per dimostrare questa seconda affermazione:

\begin{equation*}
    \frac{dW}{dt} = -\frac{1}{\mu_0}\oint_{\partial V} \left(\bm E \times \bm B\right) d\bm a - \pdv{t}\frac{1}{2}\int_V \left(\frac{1}{\mu_0}B^2- \varepsilon_0 {E^2}\right) dV
\end{equation*}

Definisco il \emph{vettore di Poynting}
\begin{equation}
    \bm S = \frac{1}{\mu_0} \bm E \times \bm B
\end{equation}

La potenza diventa:
\begin{equation}
    \frac{dW}{dt}= - \oint_{\partial V} \bm S \cdot d \bm a - \pdv{U_EM}{t}
\end{equation}



\end{document}