\documentclass[a4paper]{scrarticle}

\usepackage[T1]{fontenc}
\usepackage[utf8]{inputenc}
\usepackage[italian]{babel}

\usepackage{amsmath, amsthm, amsfonts}
\usepackage{bm}
\usepackage{array}
\usepackage{physics}
\usepackage{rotating}
\usepackage{booktabs}

\begin{document}
    
\section{Equazioni di Maxwell generali}

\[
\begin{array}{ll}
    \toprule
    \text{Forma Locale} & \text {forma Globale} \\
    \midrule
    \displaystyle \div{\va D} = \rho_{free}             & \displaystyle \varPhi(D)   = \oint_S \va D \cdot \, d\Sigma= Q_{free}\\
    \displaystyle \div{\va B} = 0                       & \displaystyle \varPhi(B)   = \oint_S \va B \cdot \, d\Sigma= 0\\
    \displaystyle \curl{\va E}= -\pdv{\va B}{t}         & \displaystyle \varGamma(E) = \oint_\gamma \va E \cdot \,dl = -\dv{\varPhi(\va B)}{t}\\
    \displaystyle \curl{\va H}= \va J +  \pdv{\va D}{t} & \displaystyle \varGamma(H) = \oint_\gamma \va H \cdot \,dl = \int_S \va J \cdot \,dS + \dv{\varPhi( \va D)}{t}\\
    \bottomrule
\end{array}    
\]

\subsection{Significati}

\begin{itemize}
    \item I Equazione: l'oggetto che genera il campo $\bm D$ sono le cariche libere.
    \item II Equazione: In assenza di campo magnetico variabile il campo elettrico è conservativo. Posso definire un potenziale Scalare.
    \item III Equazione: Non esiste un monopolo magnetico.
    \item IV Equazione: Il campo magnetico è solenoidale. Ha un potenziale vettore.
\end{itemize}

\section{Condizioni al contorno}

Il contorno è la superficie di separazione da un mezzo all'altro.

\[
\begin{array}{lll}
    \toprule
    & \text{mezzo lineare} & \sigma_f = 0\quad K_f = 0\\
    \midrule
    D_{1\perp} - D_{2\perp} = \sigma_f & \varepsilon_1 E_{1\perp} - \varepsilon_2 E_{2\perp} & \varepsilon_1 E_{1\perp} = \varepsilon_2 E_{2\perp}\\
    \bm E_{1\parallel} = \bm E_{2\parallel} & \bm E_{1\parallel} = \bm E_{2\parallel} & \bm E_{1\parallel} = \bm E_{2\parallel} \\
    \bm H_{1\parallel} - \bm H_{2\parallel} = \bm K_f \times \va n  & \frac{1}{\mu_1}\bm B_{1\parallel} - \frac{1}{\mu_2}\bm B_{2\parallel} = \bm K_f \times \va n & \frac{1}{\mu_1}\bm B_{1\parallel} = \frac{1}{\mu_2}\bm B_{2\parallel} \\
    B_{1\perp} = B_{2\perp} & B_{1\perp} = B_{2\perp} & B_{1\perp} = B_{2\perp} \\
    \bottomrule
\end{array}   
\]

Dove $K_f$ è una densità di corrente superficiale, mentre $\sigma_f$ è una densità di carica superficiale.

\section{Relazioni tra i vari campi}

\subsection{Campi}

\begin{gather}
    \va D = \varepsilon_0 \va E + \va P\\
    \va P = \chi\varepsilon_0 \va E \\
    \va D = \varepsilon_0 \varepsilon_r \va E \quad \text{Nei materiali isotropi, omogenei e lineari}\\
    \va H = \va B/\mu_0 - \va M
\end{gather}

\subsection{Relazione campi con sorgenti}
\[
\begin{array}{cc}
    \div {\va P} = - \rho_{bounded} &
    \va P \cdot  \vu n = \sigma_{bounded}\\
    \div {\va D} = \rho_{free} &
    \va D \cdot  \vu n = \sigma_{free}\\
    \curl {\va M} = \va {J_{Amp, V}} &
    \va M \times \vu n = \va {J_{Amp, S}}
\end{array}[cc]
\]


\subsection{Potenziali}

\begin{gather}
    \va B = \curl{\va A} \\
    \curl{\va E} = -\dv{t}\curl{\va A}\\
    -\grad{V} = \va E + \pdv{\va A}{t}
\end{gather}

\subsection{Formule Particolari}

\subsubsection*{Legge di Coulomb}

\begin{equation}
    \bm F_{12} = \frac{1}{4\pi\varepsilon_0}\frac{q_1 q_2}{|r_1 - r_2|^3}(\bm r_1 - \bm r_2)
\end{equation}

La forza di Coulomb è lineare, vale pertanto il \emph{Principio di Sovrapposizione}, quindi per trovare la forza su una carica devo sovrapporre gli effetti delle altre cariche agenti.

\subsection*{Distribuzione continua in un volume di carica}

\begin{equation}
    \bm E(\bm r_0 ) = \int_{V} \frac{\rho(\bm r) \text{d}^3 \bm r}{|\bm r_0 - \bm r|^2} \hat{u}(\bm r)
\end{equation}

\subsubsection*{Definizione di Potenziale}
(Si ottiene da $E = -\nabla V$):

\begin{equation}
    V(r_A) - V(r_B) = \int_A^B \bm E \cdot ds
\end{equation}

\subsubsection*{Definizione Generale di Potenziale}

\begin{equation}
    \phi_p(\bm r) = \frac{1}{4\pi\varepsilon_0}\int_{V}\frac{\rho(\bm r') d^3\bm r'}{\left|\bm r - \bm r'\right|}
\end{equation}


\subsubsection*{Equazione di Laplace}

Si ottiene combinando la definizone di campo elettrico mediante il potenziale e la I equazione di Maxwell per il campo elettrico:

\begin{gather*}
    \div E = \frac{\rho}{\varepsilon_0}\\
    E = - \grad \phi\\
    - \div \grad \phi = \frac{\rho}{\varepsilon_0}\\
    \laplacian \phi = -\frac{\rho}{\varepsilon_0} \quad \text{Equazione di Poisson}
\end{gather*}

\subsubsection*{Equazione di Poisson}

è l'omogenea dell'equazione di Laplace. Le due coincidono quando $\rho = 0$

\begin{equation}
    \laplacian \phi = 0
\end{equation}

La soluzione dell'equazione di Laplace è unica ed è data dalla somma della soluzione omogenea alla soluzione di Poisson, unitamente alle condizioni al contorno che ne danno l'unicità.

\subsubsection*{Capacità di un conduttore}

\begin{equation}
    C = \frac{q}{V} = 
    \frac{\displaystyle \oint_\Sigma \sigma (x',y',z') d\Sigma}{\displaystyle \frac{1}{4\pi\varepsilon_0} \oint_\Sigma \frac{\sigma(x',y',z')}{\left| \bm{r} - \bm{r'} \right|}} d\Sigma
\end{equation}

\subsubsection*{Condensatore a Facce Piane Parallele}
La Capacità dipende solo dalla geometria del sistema
\begin{equation*}
    C = \varepsilon_0 \frac{A}{d}
\end{equation*}


\subsubsection*{Legge di Ampere}
Forza tra due fili paralleli di lunghezza $l$ percorsi da corrente $i$, a distanza $a$:
\begin{equation*}
    F = \mu_0  \frac{i_1i_2}{2\pi a}l
\end{equation*}


\subsubsection*{Legge di Biot Savart} 
(Campo Magnetico $\bm B$  generato da un circuito $\mathcal{C}$ percorso da una corrente $i$):

\begin{equation}
    \frac{\mu_0 i}{4 \pi} \oint_{\mathcal{C}} \frac {d\bm l \times \bm r}{r^3}
\end{equation}

\subsubsection*{Forza di Lorentz}

\begin{equation}
   \bm F = q (\bm E + \bm v \times \bm B )
\end{equation}

\subsubsection*{Campi Magnetici Famosi}
\begin{itemize}
    
    \item Filo infinito:
    \begin{equation}
        \bm B_0(\bm r) = \frac{\mu_0 I}{2\pi r} \bm \theta
    \end{equation}
    
    \item Solenoide rettilineo lungo:
    \begin{equation}
        \bm B_0 (\bm r) = \frac{\mu_0NI}{l} \bm z= \mu_0 nI \bm z
    \end{equation}
    
    \item Asse di spira circolare:
    \begin{equation}
        \bm B_0 (\bm r) = \frac{\mu_0}{2\pi} \frac{I \bm S}{(R^2 + z^2)^{3/2}}
    \end{equation}
    
    \item Nastro di corrente:
    \begin{equation}
        \bm B_0(\bm x) = \bm n \frac{\mu_0 I}{2\pi b} \ln\left(\frac{l + b}{l}\right)
    \end{equation}
\end{itemize}

\subsubsection*{Autoinduttanza}
\begin{equation}
    \varPhi = L i
\end{equation}
\begin{equation}
    fem = -L \frac{dI}{dt}
\end{equation}
\subsubsection*{Mutua Induttanza} 
\begin{equation}
    \varPhi_{k, i} = M_{k , i} \cdot i_k 
\end{equation}
\begin{equation}
    fem = -M_{k, i} \frac{di_k}{dt}
\end{equation}

Induttanza solenoide
\begin{equation*}
    L = \mu_0 \mu_r {N^2} S
\end{equation*}


\subsection{Onde}

\begin{align}
    \va S & = \frac{\bm E \times \bm B}{\mu_0 \mu_r}\\
    I & = <S> = \frac{1}{2}\varepsilon v E^2_0
\end{align}

Formula per assorbimento energia:

\begin{align}
    \mathcal E =&  \varepsilon_r \varepsilon_0 E_0^2\quad &\text {materiali assorbenti}\\
    \mathcal E =& 2\varepsilon_r \varepsilon_0 E_0^2\quad &\text {materiali riflettenti}
\end{align}

\pagebreak

\begin{sidewaystable}
\caption{Formule differenziali necessarie per esame}
    \[
\begin{array}{cccc}
    \toprule
    \text{nome} & \text{Cartesiane} & \text{Cilindriche} & \text{Sferiche}\\
\midrule
% Formule per Gradiente
\displaystyle \grad{f} & 
    % Cartesiane
    \displaystyle \pdv{f}{x} \va x + \pdv{f}{y} \va y + \pdv{f}{z} \va z &
    % Cilindriche
    \displaystyle \pdv{f}{\rho} \va {\bm \rho} + \frac{1}{\rho}\pdv{f}{\phi}\va {\bm\phi} + \pdv{f}{z} \va z &
    % Sferiche
    \displaystyle \pdv{f}{r} \va r + \frac{1}{r} \pdv{f}{\theta} \va {\bm\theta} + \frac{1}{r\sin{\theta}} \pdv{f}{\phi} \va{\bm \phi}\\ 
\midrule
% Formule per Divergenza
\displaystyle \div{\bm{F}} &
    % Cartesiane
    \displaystyle \pdv{f}{x} + \pdv{f}{y} + \pdv{f}{z} &
    % Cilindriche
    \displaystyle \frac{1}{\rho} \pdv{ \rho A_\rho } { \rho}
    + \frac{1} {\rho}\pdv {A_\phi} {\phi}
    + \pdv{ A_z} {z}&
    % Sferiche
    \displaystyle \frac{1}{r^2} \pdv{r^2 F_r}{r} + \frac{1}{r \sin \theta} \pdv{F_{\theta} \sin \theta}{\theta} + \frac{1}{ r \sin \theta}\pdv{F_{\phi}}{\phi}\\ 
\midrule
% Formule per Rotore
\displaystyle \curl{\bm{F}} & 
    % Cartesiane
\begin{matrix}
    \displaystyle \bigg(\pdv{F_z}{y} - \pdv{F_y}{z} \bigg) \boldsymbol{\hat x}\\
    \displaystyle \bigg(\pdv{F_x}{z} - \pdv{F_z}{x}\bigg) \boldsymbol{\hat y}\\
    \displaystyle \bigg(\pdv{F_y}{x} - \pdv{F_x}{y}\bigg) \boldsymbol{\hat z}
\end{matrix} & 
    % Cilindriche
\begin{matrix}
    \displaystyle \bigg (\frac{1 }{ \rho}\pdv{ A_z }{ \phi}
    - \pdv{A_\phi }{  z} \bigg ) \boldsymbol{\hat \rho}  \\
    \displaystyle \bigg (\pdv{A_\rho }{z} - \pdv{A_z }{\rho} \bigg) \boldsymbol{\hat \phi} \\
    \displaystyle \frac{1 }{ \rho}\bigg (\pdv{( \rho A_\phi ) }{\rho}
    - \pdv{A_\rho}{\phi} \bigg ) \boldsymbol{\hat z}
\end{matrix} &
    % Sferiche
\begin{matrix}
    \displaystyle \frac{1}{ r\sin\theta} \bigg (\frac{\partial }{ \partial \theta} ( A_\phi\sin\theta )
    - \frac{\partial A_\theta }{ \partial \phi} \bigg ) \boldsymbol{\hat r} \\
    \displaystyle \frac{1}{ r} \bigg (\frac{1 }{ \sin\theta}\pdv{A_r }{\phi}
    - \pdv{( r A_\phi )}{r}  \bigg ) \boldsymbol{\hat \theta}\\
    \displaystyle \frac{1 }{ r}\bigg (\frac{\partial }{ \partial r} ( r A_\theta )
    - \pdv{ A_r }{\theta} \bigg) \boldsymbol{\hat \phi}
\end{matrix}
\\
\midrule
% Formule per Laplaciano Scalare
\nabla^2 f &
    % Cartesiane
    \frac{\partial^2 f }{ \partial x^2} + \frac{\partial^2 f }{ \partial y^2} + \frac{\partial^2 f }{ \partial z^2}&
    % Cilindriche
    \frac{1 }{ \rho}\frac{\partial }{ \partial \rho} \bigg (\rho \frac{\partial f }{ \partial \rho} \bigg )
    + \frac{1 }{ \rho^2}\frac{\partial^2 f }{ \partial \phi^2}
    + \frac{\partial^2 f }{ \partial z^2}
    &
    % Sferiche
    \frac{1 }{ r^2}\frac{\partial }{ \partial r} \bigg (r^2 \frac{\partial f }{ \partial r} \bigg)
    + \frac{1 }{ r^2\sin\theta}\frac{\partial }{ \partial \theta} \bigg(\sin\theta \frac{\partial f }{ \partial \theta} \bigg)
    + \frac{1 }{ r^2\sin^2\theta}\frac{\partial^2 f }{ \partial \phi^2}\\
\midrule
% Formule per laplaciano Vettoriale
\nabla^2 \mathbf{A} &
    % Cartesiano
    \nabla^2 A_x \mathbf{\hat x} + \nabla^2 A_y \mathbf{\hat y} + \nabla^2 A_z \mathbf{\hat z} & 
    % Cilindriche
\begin{matrix}
    \displaystyle \bigg (\nabla^2 A_\rho - \frac{A_\rho }{ \rho^2}
    - \frac{2 }{ \rho^2}\frac{\partial A_\phi }{ \partial \phi} \bigg) \boldsymbol{\hat\rho}\\
    \displaystyle \bigg (\nabla^2 A_\phi - \frac{A_\phi }{ \rho^2}
    + \frac{2 }{ \rho^2}\frac{\partial A_\rho }{ \partial \phi} \bigg ) \boldsymbol{\hat\phi}\\
    \displaystyle (\nabla^2 A_z ) \boldsymbol{\hat z}
\end{matrix} & 
    % Sferiche
\begin{matrix}
    \bigg (\nabla^2 A_r - \frac{2 A_r }{ r^2}
    - \frac{2 }{ r^2\sin\theta}\frac{\partial (A_\theta \sin\theta) }{ \partial\theta}
    - \frac{2 }{ r^2\sin\theta}\frac{\partial A_\phi }{ \partial \phi} \bigg ) \boldsymbol{\hat r}\\
    \bigg (\nabla^2 A_\theta - \frac{A_\theta }{ r^2\sin^2\theta}
    + \frac{2 }{ r^2}\frac{\partial A_r }{ \partial \theta}
    - \frac{2 \cos\theta }{ r^2\sin^2\theta}\frac{\partial A_\phi }{ \partial \phi} \bigg ) \boldsymbol{\hat\theta}\\
    \bigg (\nabla^2 A_\phi - \frac{A_\phi }{ r^2\sin^2\theta}
    + \frac{2 }{ r^2\sin\theta}\pdv{ A_r }{  \phi}
    + \frac{2 \cos\theta }{ r^2\sin^2\theta}\frac{\partial A_\theta }{ \partial \phi} \bigg ) \boldsymbol{\hat\phi} & 
\end{matrix}
\\ \bottomrule
\end{array}
\]
\end{sidewaystable}


\section {Relazioni notevoli (valgono in tutti i sistemi di riferimento)}
\begin{gather}
\div  \grad f = \nabla \cdot (\nabla f) = \nabla^2 f \quad\text{Operatore di Laplace o Laplaciano}\\
\curl \grad f = \nabla \times (\nabla f) = 0\\
\div  \curl \mathbf{F} = \nabla \cdot (\nabla \times \mathbf{F}) = 0\\
\curl \curl \mathbf{F} = \nabla \times (\nabla \times \mathbf{F}) = \nabla (\nabla \cdot \mathbf{F}) - \nabla^2 \mathbf{F}\\
\nabla^2 f g = f \nabla^2 g + 2 \nabla f \cdot \nabla g + g \nabla^2 f
\end{gather}
Formula di Lagrange per prodotto vettoriale: 
\begin{gather}
\mathbf{A} \times (\mathbf{B} \times \mathbf{C}) = \mathbf{B} (\mathbf{A} \cdot \mathbf{C}) - \mathbf{C} (\mathbf{A} \cdot \mathbf{B})\\
\nabla\cdot(f \mathbf A)=f \nabla\cdot\mathbf A+\mathbf A\cdot\nabla f\\
\nabla\times f \mathbf A= f \nabla\times \mathbf A-\mathbf A\times \nabla f\\
\nabla ( \mathbf{A} \cdot \mathbf{B} ) 
  = ( \mathbf{A} \cdot \nabla ) \mathbf{B}
  + ( \mathbf{B} \cdot \nabla ) \mathbf{A}
  + \mathbf{A} \times ( \nabla \times \mathbf{B} )
  + \mathbf{B} \times ( \nabla \times \mathbf{A} )
\end{gather}
\end{document}