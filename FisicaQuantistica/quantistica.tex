\documentclass[a4paper]{scrreport}


\usepackage[T1]{fontenc}
\usepackage[utf8]{inputenc}
\usepackage[italian]{babel}

\usepackage[parts, pdfspacing]{classicthesis}

\usepackage{amsmath, amsthm, amsfonts}
\usepackage{array}
\usepackage{physics}
\usepackage{rotating}
\usepackage{booktabs}

\begin{document}

\author{Stefano Pilosio}
\title{Appunti di quantistica}

\maketitle

\tableofcontents

\chapter{Particella Libera}

\subsection{Hamiltoniana}

L'hamiltoniana che descrive il sistema è:

\begin{equation}
    H = \frac{ p^ 2}{2m}
\end{equation}

\chapter{Barriera di Potenziale}

\subsection{Hamiltoniana}

\begin{align}
    H    & = \frac{ p^ 2}{2m} + V(x)\\
    V(x) & = 
    \begin{cases}
    0   & x < 0\\
    V_0 & x \ge 0   
    \end{cases}
\end{align}

\chapter{Stati Legati}

\section{Buca di Potenziale}

\subsection{Hamiltoniana}

L'hamiltoniana che descrive il sistema è:
\begin{align}
    H  &= \frac{p^ 2}{2m} + V(x)\\
    V(x) &=
    \begin{cases}
        0 & |x|< a_0\\
        V_0 & |x|\ge a_0
    \end{cases}
\end{align}

\section{Potenziale Armonico}

\subsection{Hamiltoniana}

L'hamiltoniana che descrive il sistema è:
\begin{equation}
    H = \frac{ p^ 2}{2m} + \frac{m\omega ^ 2 x ^ 2}{2}
\end{equation}


\end{document}
