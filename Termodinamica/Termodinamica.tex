\documentclass[a4paper]{article}
% Encoding Text
\usepackage[T1]{fontenc}
\usepackage[utf8]{inputenc}
\usepackage[italian]{babel}
% Required Libraries
\usepackage{amsthm}
\usepackage{amsmath}
\usepackage{amssymb}

% Da usare in presenza di Definizione di Oggetti
\theoremstyle{definition}
\newtheorem{defin}{Def.}
% Da usare in presenza di una legge
\theoremstyle{plain}
\newtheorem{lex}{Legge}
\newtheorem{prin}{Principio}
% Per avere comandi facili da ricordare
\newcommand {\bd} [1] {\textbf{#1}}


\begin{document}
% Titolo
\author{Stefano Pilosio}
\title {Appunti Termodinamica}
\maketitle


\section{Lezione 02 Marzo 2021}
% Appunti della Lezione del 2 Marzo 2021
\subsection{Definizioni}

A seguire una lista degli elementi che saranno usati in tutto il resto del corso

\begin{defin}[sugli elementi della termodinamica]\hfill
\begin{itemize} 
    \item Si definisce \bd{Sistema} una porzione dell'universo distanta da quanto ha intorno.
    \item Si definisce \bd{Ambiente} tutto ciò che non appartiene al sistema, ma interagisce con questo.
    \item L'\bd{Universo Termodinamico} è l'unione del Sistema con l'Ambiente 
    \item L'ambiente e il sistema possono interagire tra di loro, tramite uno scambio di Energia o di Materia. 
    \item Il \bd{Confine} definisce la separazione tra Ambiente e Sistema, esso può essere di tre tipi:
    \begin{itemize}
        \item \bd{Aperto}: Permette lo scambio di Materia ed Energia.
        \item \bd{Chiuso}: Permette lo scambio solo di Energia, in questo caso la parete è  detta \emph{Diatermana}
        \item \bd{Isolato}: Non permette lo scambio, in questo caso la parete è detta \emph{Adiatermana}
    \end{itemize} 
\end{itemize}
\end{defin}

\bd{Nota:}Esistono alcune pareti flessibili, dette \bd{Pistoni},
 che permettono una variazione del volume e quindi della pressione. 

Riportiamo inoltre alcune definizioni riguardanti le proprietà degli oggetti:

\begin{defin}[sulle proprietà] Esse possono essere:
    \begin{itemize}
        \item \bd{Estensive} quano esse sono addittive (massa, Volume, \dots)
        \item \bd{Intensive} quando non sono addittive (densità, pressione, temperatura, \dots)
    \end{itemize}
\end{defin}

\subsection{Equilibrio Meccanico}

Si ottiene l'equilibrio meccanico quando $\sum_{i}\vec{F}_i=0$,
in particolare nel gaso di due gas contentuti in due volumi $V_a$ e $V_b$, separati da un pistone di superfice $S$ 
a pressione $p_a$ e $p_b$,
per ottenere l'equilibrio si deve verificare :
\begin{equation*}
    F_a = p_a \cdot S = p_b \cdot S = F_b
\end{equation*}
Questo Implica che $p_a=p_b$, se al sistema aggiungo un terzo gas con volume $V_c$ e pressione $p_c$
per ottenere l'equilibrio meccanico si deve avere $p_a=p_b=p_c$.

Da questo di desume che nel caso dei gas la pressione è un'indicatore dell'equilibrio meccanico.

\subsection{Equilibrio Termico}

% Appunti presi il 9 Marzo
\section{Lezione 09 Marzo 2021}

\subsection{Dilatazione Volumica e Lineare}

\subsection{Legge di Charles}

\subsection {Legge di Gay-Lussac}

\subsection {Legge di Boyle-Marriot}

\subsection {Il calore}
% Appunti presi l' 11 Marzo
\section{Lezione 11 Marzo 2021}

\subsection {Teoria Cinetica dei gas}

\subsection {Enetiga Interna}

\subsection {Distribuzione di Maxwell delle velocità}

% Appunti presi il 16 marzo
\section{Lezione 16 Marzo 2021}

\subsection{Equilibrio Termodinamico}

\subsection {Trasformazioni e Condizioni di Reversibilità}

\subsection{Lavoro}

\subsection {Primo Principio}

\section{Lezione 18 Marzo 2021}
%\section{Lezione Marzo 2021}
%\section{Lezione Marzo 2021}
%\section{Lezione Marzo 2021}
%\section{Lezione Marzo 2021}
%\section{Lezione Marzo 2021}


\end{document}

