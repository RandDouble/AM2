\documentclass[a4paper]{article}
\usepackage[T1]{fontenc}
\usepackage[utf8]{inputenc}
\usepackage[italian]{babel}
\usepackage{amsmath}
\usepackage{amssymb}

\newcommand{\numberset}{\mathbb}
\newcommand{\N}{\numberset{N}}
\newcommand{\R}{\numberset{R}}

\begin{document}

\author{Stefano Pilosio}
\title {Cose Utili per AM2}
\maketitle

\section{Integrali Indefiniti}

\section{Sostituzioni utili integrali definiti}

\section{Convergenza Integrali Impropri}
\begin{equation*}
    \begin{gathered}
        %caso 1
        \int_0^a{\frac{1}{x^\alpha}\text{d}x}
        \begin{cases}
            \alpha<1   \quad \text{converge}\\
            \alpha\ge1 \quad \text{non converge}
        \end{cases}
        \\
        a \in \mathbb{R} \setminus {0}
        \\
        \\
        %caso 2
        \int_a^\infty{\frac{1}{x^\alpha}\text{d}x}
        \begin{cases}
            \alpha>1   \quad \text{converge}\\
            \alpha\le1 \quad \text{non converge}
        \end{cases}
        \\
        a \in (0, \infty)
        \\
        \\
        %caso 3
        \int_0^a{\frac{1}{x^\alpha\log^{\beta} x}\text{d}x}
        \begin{cases}
            \alpha<1 \quad \forall\beta\quad \text{converge}\\
            \alpha=1 \quad \beta>1     \quad \text{converge}\\
            \alpha=1 \quad \beta\le1   \quad \text{non converge}\\
            \alpha>1 \quad \forall\beta\quad \text{non converge}
        \end{cases}
        \\
        a \in (0,1)
        \\
        \\
        %caso 4
        \int_a^\infty{\frac{1}{x^\alpha\log^{\beta} x}\text{d}x}
        \begin{cases}
            \alpha<1 \quad \forall\beta\quad \text{converge}\\
            \alpha=1 \quad \beta>1     \quad \text{converge}\\
            \alpha=1 \quad \beta\le1   \quad \text{non converge}\\
            \alpha>1 \quad \forall\beta\quad \text{non converge}
        \end{cases}
        \\
        a \in (1, \infty)
        \\
        \\
        %caso 5
        \int_1^a{\frac{1}{\log^{\beta} x}\text{d}x}
        \begin{cases}
            \beta<1   \quad \text{converge}\\
            \beta\ge1 \quad \text{non converge}\\
            \end{cases}
        \\
        a \in (0,1) \vee (1, \infty)
        \\
    \end{gathered}
\end{equation*}
\section{Approssimazioni}

\section{Lista Metodi Eq. Differenziali}

\section{Checklist Studi di Funzione}

\section {Ricerca dei massimi e dei minimi}
    \subsection{Condizioni Necessarie}
        \newtheorem{nec}{Condizioni Necessarie Estremante Libero}
        \begin{nec}    
        $\\f\colon\Omega\subseteq\R\to\R$, $ x_0\in\Omega^o$, f derivabile in $x_0$.\\
        Se $x_0$ è estremante di f $\implies \, \nabla f(x_0)=\bar{0}$ 
        \end{nec}
        Questo non basta in quanto un punto estremante in più dimensioni può essere sia un massimo, che un minimo o, infine, un punto di sella 

        Nel caso del punto di sella questo non è né un minimo né un massimo, tuttavia è stazionario 

\section{Successioni di funzioni}
\begin{enumerate}
    \item Capisci dove è definita la funzione da analizzare per comprendere la continuità della successione
    \item Fai il $\lim_{n \to \infty} f_n(x)$ per capire qual è la funzione limite
    \item Verifica che in tutti i punti si abbia convergenza puntuale alla funzione limite
    \item Verifica la convergenza uniforme con: $\sup_{x\in E}|f_n(x)-f(x)|$, questo è eseguito cercando un modo per dimostrare o confutare che in ogni punto che $|f_n-f| \to 0$ in un modo o nell'altro. 
\end{enumerate}

\end{document}